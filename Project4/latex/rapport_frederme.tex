\documentclass[aps,reprint]{revtex4-1}
% Engine-specific settings
% Detect pdftex/xetex/luatex, and load appropriate font packages.
% This is inspired by the approach in the iftex package.
% pdftex:
\ifx\pdfmatch\undefined
\else
    \usepackage[T1]{fontenc}
    \usepackage[utf8]{inputenc}
\fi
% xetex:
\ifx\XeTeXinterchartoks\undefined
\else
    \usepackage{fontspec}
    \defaultfontfeatures{Ligatures=TeX}
\fi
% luatex:
\ifx\directlua\undefined
\else
    \usepackage{fontspec}
\fi
% End engine-specific settings
\usepackage[english]{babel}
\usepackage{csquotes}
% \usepackage[backend=biber, sortcites]{biblatex}
\usepackage{url}
\usepackage{textcomp}
\usepackage[usenames,dvipsnames,svgnames, table]{xcolor}
\usepackage[font={scriptsize}]{caption}
\usepackage{amsmath} \usepackage{amsthm} \usepackage{amsfonts}
\usepackage{amssymb}
\usepackage{enumerate}
\usepackage{tikz} \usepackage{float}
\usepackage[procnames]{listings}
\usepackage{pstool} \usepackage{pgfplots}
\usepackage{wrapfig} \usepackage{graphicx} \usepackage{epstopdf}
\usepackage{afterpage}
\usepackage{physics}
\usepackage{multirow}
\usepackage{gensymb}
\usepackage{algorithm}
\usepackage{microtype}
\usepackage[noend]{algpseudocode}
\usepackage{xcolor,colortbl}
\usepackage{microtype}
\usepackage{geometry}
\usepackage{hyperref}
\usepackage{graphicx}
\usepackage{caption}
\usepackage{subcaption}
\usepackage{lipsum}
% \usepackage{pythontex}
% \usepackage{authblk}
\usepackage{nth}
\usepackage{siunitx}
% \usepackage[toc,page]{appendix}
\floatstyle{plaintop}
\restylefloat{table}

% Custom commands
\newcommand{\unit}[1]{\:\mathrm{#1}}
\newcommand{\noref}[1]{\hyperref[#1]{\ref*{#1}}}
\newcommand{\nonref}[1]{\hyperref[]{\ref*{#1}}}
\newcommand\blankpage{%
  \null
  \thispagestyle{empty}%
  \addtocounter{page}{-1}%
  \newpage}
\newcommand{\mean}[1]{\langle #1 \rangle}

% Default fixed font does not support bold face
\DeclareFixedFont{\ttb}{T1}{txtt}{bx}{n}{7} % for bold
\DeclareFixedFont{\ttm}{T1}{txtt}{m}{n}{7}  % for normal

\newcommand\numberthis{\addtocounter{equation}{1}\tag{\theequation}}
\DeclareCaptionFont{white}{\color{white}}
\DeclareCaptionFormat{listing}{\colorbox{gray}{\parbox{\columnwidth}{#1#2#3}}}
\pgfplotsset{compat=1.14} %TODO: Setting this removed several error messages, should it be here!?


% Biber for references
% \bibliographystyle{aipauth4-1}

\begin{document}
\sisetup{detect-all}
\title{Ising (Or should I say Ezing?)}
\author{Frederik J. Mellbye}
\affiliation{University of Oslo, Oslo, Norway \\ Source code available at: \url{https://github.com/Caronthir/FYS3150/tree/master/Project4}}
\date{\today}

\begin{abstract}
Once again the abstract abstractifies the abstractness of the abstract mathematical
equations than govern the universe.
\end{abstract}
\maketitle
\tableofcontents
\makeatletter
\let\toc@pre\relax
\let\toc@post\relax
\makeatother

\newpage

\section{Introduction}
\label{sec:introduction}
The Ising model is a widely used model, because of a wide range of applications,
for instance in studies of phase transitions to simulations in statistics.
\section{Theory}
\label{sec:theory}
\subsection{The Ising model}
A square lattice of $N \times N$ binary magnetic dipoles (quantified atomic spins)
make up the Ising model. The dipoles can only have spins $s_i$ up ($1$) and down ($-1$).
In the Ising model the temperature $T$ is held constant.

The energy is given by
\begin{align}
  E = -J \sum_{\langle kl \rangle}^{N} s_k s_l
\end{align}
where for each lattice location the sum is taken over the neighbors only. $J$ is
a constant that determines the strength of the interaction between the nodes
(The complexity of the quantum mechanics in such a system is baked into
this constant). It is assumed that there is a ferromagnetic ordering so that $J > 0$.

The magnetization is simply given as the sum of the spins:
\begin{align}
  \mathcal{M} = \sum_i s_i
\end{align}

The thermodynamical quantities that are computed in this paper can be constructed
from the above expressions. The heat capacity under constant volume $C_V$ is given
by
\begin{align}
  C_V = \frac{\mean{E^2} - \mean{E}^2}{kT^2}
\end{align}
and the magnetic suceptibility is similarly given by
\begin{align}
  \chi = \frac{\mean{\mathcal{M}^2} - \mean{\mathcal{M}}^2}{kT}
\end{align}

\subsection{Boundary conditions}
There are several ways to model the boundaries, which have different advantages
and disadvantages. Perhaps the easiest solution is for the boundary spins to have
no outside neighbors. Usually real materials have an enormous amount of spins,
and the effect of the boundaries is negligible. In the Ising model an infinite
number of spins is not possible, and for small lattice sizes an appropriate
method needs to be used. For the intents and purposes of this paper, an ideal
method is to have so-called periodic boundaries. This involves treating the
spins on the oppsite side of the lattice as the neighbors of the boundary spins.
This is a much better approximation to an infinitely large lattice, because
each spin has access to four neighbors.

\subsection{Metropolis algorithm}

\subsection{Phase transitions in the Ising model}

\subsection{Analytic solution in the 2 x 2 lattice case}
\begin{figure}[H]
  \centering
  \begin{tikzpicture}[thick]
    \draw[->] (0,0) -- (0,.3)  node [label=left:{s1}] {};
    \draw[->] (1,0) -- (1,.3)  node [label=right:{s2}] {};
    \draw[->] (0,1) -- (0,1.3) node [label=left:{s3}] {};
    \draw[->] (1,1) -- (1,1.3) node [label=right:{s4}] {};
  \end{tikzpicture}
  \caption{2 x 2 lattice with all spins pointing up and an example of enumerating
  the lattice sites.}
  \label{fig:22lattice}
\end{figure}
\section{Method}
\label{sec:method}
\subsection{Energy calculations}

\subsection{Periodic boundaries}
In this paper, the boundaries are held open to mimic real-world behaviour and
thus acheive better results. A clever method was used to implement this, which
is available in the source code.

\subsection{Metropolis algorithm}

\subsection{Phase transitions}

\subsection{Testing}
\section{Results}
\label{sec:results}

\section{Discussion}
\label{sec:discussion}

\section{Conclusion}
\label{sec:conclusion}

\bibliography{references}
\blankpage
\appendix
\section{Appending appendices is easy with the appendix package.}
\blankpage
\end{document}

% Local Variables:
% TeX-engine: luatex
% End:
